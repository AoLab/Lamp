
%%%%%%%%%%%%%%%%%%%%%%%%%%%%%%%%%%%%%%%%%
% University/School Laboratory Report
% LaTeX Template
% Version 3.1 (25/3/14)
%
% This template has been downloaded from:
% http://www.LaTeXTemplates.com
%
% Original author:
% Linux and Unix Users Group at Virginia Tech Wiki
% (https://vtluug.org/wiki/Example_LaTeX_chem_lab_report)
%
% License:
% CC BY-NC-SA 3.0 (http://creativecommons.org/licenses/by-nc-sa/3.0/)
%
%%%%%%%%%%%%%%%%%%%%%%%%%%%%%%%%%%%%%%%%%

%----------------------------------------------------------------------------------------
%saPACKAGES AND DOCUMENT CONFIGURATIONS
%----------------------------------------------------------------------------------------

\documentclass{article}

\usepackage{graphicx} % Required for the inclusion of images

\setlength\parindent{0pt} % Removes all indentation from paragraphs

\renewcommand{\labelenumi}{\alph{enumi}.} % Make numbering in the enumerate environment by letter rather than number (e.g. section 6)

%\usepackage{times} % Uncomment to use the Times New Roman font

%----------------------------------------------------------------------------------------
%fontDOCUMENT INFORMATION
%----------------------------------------------------------------------------------------

\title{In the name of God \\ IoT Lab \\ Spring 2016} % Title

\author{Parham \textsc{Alvani} \\ Iman \textsc{Tabrizian}} % Author name

\date{\today} % Date for the report

\begin{document}

\maketitle % Insert the title, author and date

\begin{center}
\begin{tabular}{l r}
    Date Performed: & Esfand 24, 2016 \\ % Date the experiment was performed
\end{tabular}
\end{center}

% If you wish to include an abstract, uncomment the lines below
% \begin{abstract}
% Abstract text
% \end{abstract}

%----------------------------------------------------------------------------------------
%abstractSECTION 1
%----------------------------------------------------------------------------------------

\section{Report}

Today we tested the lamp provided by the electrical engineering students together
with the micocontroller and successfuly turn it on by \texttt{screen}, as mentioned
in the src files the baud rate must be \textbf{9600} and this also wasted some
of our time. Today we failed to turn the lamp on using the KAA server notification.


\end{document}


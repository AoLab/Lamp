
%%%%%%%%%%%%%%%%%%%%%%%%%%%%%%%%%%%%%%%%%
% University/School Laboratory Report
% LaTeX Template
% Version 3.1 (25/3/14)
%
% This template has been downloaded from:
% http://www.LaTeXTemplates.com
%
% Original author:
% Linux and Unix Users Group at Virginia Tech Wiki
% (https://vtluug.org/wiki/Example_LaTeX_chem_lab_report)
%
% License:
% CC BY-NC-SA 3.0 (http://creativecommons.org/licenses/by-nc-sa/3.0/)
%
%%%%%%%%%%%%%%%%%%%%%%%%%%%%%%%%%%%%%%%%%

%----------------------------------------------------------------------------------------
% PACKAGES AND DOCUMENT CONFIGURATIONS
%----------------------------------------------------------------------------------------

\documentclass{article}

\usepackage{graphicx} % Required for the inclusion of images

\setlength\parindent{0pt} % Removes all indentation from paragraphs

\renewcommand{\labelenumi}{\alph{enumi}.} % Make numbering in the enumerate environment by letter rather than number (e.g. section 6)

%\usepackage{times} % Uncomment to use the Times New Roman font

%----------------------------------------------------------------------------------------
% DOCUMENT INFORMATION
%----------------------------------------------------------------------------------------

\begin{document}

\begin{titlepage}
	\centering
	\begin{tabular}{l r}
		\includegraphics[width=0.3\textwidth]{../AUT-CEIT-01.png}
		&
		\includegraphics[width=0.3\textwidth]{../AUT-CEIT-02.png}
	\end{tabular}
	\vspace{1cm}\par
	{\scshape\LARGE Amirkabir University of Technology \par}
	\vspace{1cm}
	{\scshape\Large IoT Project Report\par}
	\vspace{2cm}
	{\Large\itshape Parham Alvani\par}
	{\Large\itshape Iman Tabrizian\par}
	\vfill
	supervised by\par
	Dr.Bahador \textsc{Bakhshi}\par
    	Date Performed: Esfand 20, 1394
	\vfill
	{\large \today\par}
\end{titlepage}

%----------------------------------------------------------------------------------------
% REPORT SECTION
%----------------------------------------------------------------------------------------

\section*{Report}
Today we tested the lamp provided by the electrical engineering students together
with the micocontroller and successfuly turn it on by \texttt{screen}, as mentioned
in the src files the baud rate must be \textbf{9600} and this also wasted some
of our time. Today we failed to turn the lamp on using the KAA server notification.

\end{document}

